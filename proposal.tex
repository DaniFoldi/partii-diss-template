% Note: this file can be compiled on its own, but is also included by
% diss.tex (using the docmute.sty package to ignore the preamble)
\documentclass[12pt,a4paper,twoside]{article}
\usepackage[pdfborder={0 0 0}]{hyperref}
\usepackage[margin=25mm]{geometry}
\usepackage{graphicx}
\usepackage{parskip}
\begin{document}

\begin{center}
\Large
Computer Science Tripos -- Part II -- Project Proposal\\[4mm]
\LARGE
How to write a dissertation in \LaTeX\\[4mm]

\large
M.~Richards, St John's College

Originator: Dr M.~Richards

14 October 2011
\end{center}

\vspace{5mm}

\textbf{Project Supervisor:} Dr M.~Richards

\textbf{Director of Studies:} Dr M.~Richards

\textbf{Project Overseers:} Dr F.~H.~King  \& Dr A.~W.~Moore

% Main document

\section*{Introduction}

\emph{The problem to be addressed.}

Many students write their CST dissertations in \LaTeX\ -- and spend a
fair amount of time learning just how to do that. The purpose of this
project is to write a demonstration dissertation that provides
a starting point to show how it is done.

This core proposal document will be augmented by a separately-printed
cover sheet at the front and a resource form at the end. Additional
sheets for risk assessment and human resources may also need to be
included.

This document will elaborate much of the material that is summarised on
the additional sheets.

\section*{Starting point}

\emph{Describe existing state of the art, previous work in this area,
  libraries and databases to be used. Describe the state of any
  existing codebase that is to be built on.}

I am already able to write prose using the English language. I have an
online dictionary, etc.

\section*{Resources required}

\emph{A note of the resources required and confirmation of access.}

For this project I shall mainly use my own quad-core computer that
runs Fedora Linux. Backup will be to github and/or to an SVN
repository on an external hard disk that is dumped to writable CD/DVD
media. I have another similar computer to hand should my main machine
suddenly fail. I require no other special resources.

\section*{Work to be done}

\emph{Describe the technical work.}

The project breaks down into the following sub-projects:

\begin{enumerate}

\item The construction of a skeleton dissertation with the required
  structure. This involves writing the Makefile and making dummy
  files for the title page, the proforma, chapters 1 to 5, the
  appendices and the proposal.

\item Filling in the details required in the cover page and proforma.

\item Writing the contents of chapters 1 to 5, including examples of
  common \LaTeX\ constructs.

\item Adding a example of how to use floating figures and ``encapsulated
  PostScript'' or PDF diagrams.

\end{enumerate}

\section*{Success citeria}

\emph{Describe what you expect to be able to demonstrate at the
end of the project and how you are going to evaluate your achievement.}

The project will be a success if I have a completed dissertation with
the correct chapter titles and I have achieved my other success
criteria, which are to blah \ldots


\section*{Possible extensions}

{\em Potential further envisaged evaluation metrics or extensions.}

If I achieve my main result early I shall try the following
alternative experiment or method of evaluation \ldots


\section*{Timetable}

\emph{A workplan of perhaps ten or so two-week work-packages,
as well as milestones to be achieved along the way. Provide a
target date for each milestone.}

Planned starting date is 16/10/2011.

\begin{enumerate}

\item \textbf{Michaelmas weeks 2--4} Learn to use X. Read book Y. Read papers Z.

\item \textbf{Michaelmas weeks 5--6} Do preliminary test of Q.

\item \textbf{Michaelmas weeks 7--8} Start implementation of main task A.

\item \textbf{Michaelmas vacation} Finish A and start main task B.

\item \textbf{Lent weeks 0--2} Write progress report. Generate corpus of
  test examples. Finish task B.

\item \textbf{Lent weeks 3--5} Run main experiments and achieve working project.

\item \textbf{Lent weeks 6--8} Second main deliverable here.

\item \textbf{Easter vacation:} Extensions and writing dissertation main
  chapters.

\item \textbf{Easter term 0--2:}  Further evaluation and complete dissertation.

\item \textbf{Easter term 3:} Proof reading and then an early submission
  so as to concentrate on examination revision.

\end{enumerate}

\end{document}